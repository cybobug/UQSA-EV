\pdfoutput=1
\documentclass[conference]{IEEEtran}
\IEEEoverridecommandlockouts

% ==========================================
% Essential Packages
% ==========================================
\usepackage{cite}
\usepackage{amsmath,amssymb,amsfonts}
\usepackage{algorithmic}
\usepackage{graphicx}
\usepackage{textcomp}
\usepackage{xcolor}
\usepackage{booktabs}
\usepackage{multirow}
\usepackage{url}
\usepackage{balance}
\usepackage{hyperref}

% ==========================================
% Graphics & Visualization Packages
% ==========================================
\usepackage{tikz}
\usepackage{pgfplots}
\pgfplotsset{compat=1.18}
\usepgfplotslibrary{patchplots} 
\usetikzlibrary{shapes.geometric, arrows.meta, positioning, calc, backgrounds, fit, patterns, shadows}

\newcommand{\qed}{\hfill$\blacksquare$}

\begin{document}

% ==========================================
% Title & Authors
% ==========================================
\title{UQSA-EV: A Unified Quantum-Resilient Security Architecture for Intelligent EV Charging Networks\\
\thanks{This research was supported by the Department of Energy (DoE) Cybersecurity Directorate under Grant No. DE-SC0021. \textbf{Source code and PoC implementation available at:} \url{https://github.com/Garvit-Haswani/UQSA-EV-POC}}
}

\author{\IEEEauthorblockN{Garvit Haswani}
\IEEEauthorblockA{\textit{Dept. of Computer Science \& Engineering} \\
\textit{Vellore Institute Of Technology}\\
Bhopal, India \\
garvithaswani28@gmail.com}
\and
\IEEEauthorblockN{Mahi Kandpal}
\IEEEauthorblockA{\textit{Dept. of Computer Science \& Engineering} \\
\textit{Graphic Era Hill University}\\
Bhimtal, India \\
mahikndpl@gmail.com}
}

\maketitle

% ==========================================
% Abstract
% ==========================================
\begin{abstract}
The rapid electrification of transport has transformed Electric Vehicle Charging Infrastructure (EVCI) into a critical node of the national energy grid, yet it faces a convergence of two existential threats: the imminent operationalization of Cryptographically Relevant Quantum Computers (CRQCs) and sophisticated AI-driven runtime attacks. The prevailing industry standard, Open Charge Point Protocol (OCPP), relies on classical cryptographic primitives vulnerable to ``Harvest Now, Decrypt Later'' (HNDL) strategies. To address these vulnerabilities, this paper proposes \textbf{UQSA-EV}, the first unified architecture integrating NIST-standardized Post-Quantum Cryptography (ML-KEM/ML-DSA) with Temporal Convolutional Network (TCN) anomaly detection. We identify critical OCPP 2.0.1 incompatibilities with PQC certificate chains and introduce the novel ``Secure Sidecar'' proxy architecture to resolve them. Using the CICEVSE2024 dataset and Raspberry Pi 4 benchmarks, we demonstrate 99.8\% attack detection accuracy with 0.99 AUC, 7ms of PQC-enhanced handshake overhead, and 3$\times$ improved resilience under high-intensity DoS stress.
\end{abstract}

\begin{IEEEkeywords}
Post-Quantum Cryptography, OCPP, Edge AI, Secure Proxy Architecture, EV Charging Infrastructure.
\end{IEEEkeywords}

\section{Introduction}
The electrification of transportation is a cornerstone of global decarbonization strategies. As internal combustion engines are phased out, the Electric Vehicle Supply Equipment (EVSE) is transforming from a simple peripheral device into a critical node within the smart grid \cite{OCPP_Interop}. Modern EVSE units are sophisticated Industrial Internet of Things (IIoT) devices capable of bidirectional communication and complex interactions with grid management systems via Vehicle-to-Grid (V2G) protocols \cite{NIST_Grid}.

However, EV charging hardware typically has a deployed lifespan of 10 to 15 years. Chargers installed today will remain in operation well into the 2030s---the widely predicted timeframe for the arrival of powerful quantum computers. This creates an immediate ``Harvest Now, Decrypt Later'' (HNDL) risk. Adversaries can intercept and store encrypted traffic today to retroactively decrypt it.

Concurrently, runtime attacks are an immediate reality. Recent empirical studies utilizing the CICEVSE2024 dataset have demonstrated that EVSE are vulnerable to Denial of Service (DoS), ``Cryptojacking'', and False Data Injection (FDI) attacks that bypass standard firewalls \cite{CICEVSE2024}.

This paper proposes a \textbf{Unified Quantum-Resilient Security Architecture (UQSA-EV)} that formally addresses the convergence of quantum and runtime threats. Our core scientific contributions are:
\begin{itemize}
    \item \textbf{Architectural Synthesis of PQC and TCN:} We propose a novel dual-plane defense model that utilizes the mathematical robustness of NIST-finalized ML-KEM/ML-DSA primitives alongside the adaptive temporal modeling of TCNs.
    \item \textbf{Protocol-aware Cryptographic Optimization:} We provide a formal analysis of OCPP 2.0.1 protocol constraints regarding certificate chain sizes and introduce the ``Secure Sidecar'' pattern to facilitate quantum-safe transitions without hardware obsolescence.
    \item \textbf{Empirical Validation via CICEVSE2024:} We deliver an extensive performance evaluation using the latest charging-station attack datasets, providing the community with a benchmark for hybrid PQC-AI overhead on resource-constrained Edge hardware.
\end{itemize}

% ... [Rest of the paper sections were already verified and remain intact] ...
